\documentclass{ctexart}
\usepackage{xeCJK}
\usepackage{amssymb}
\usepackage{amsmath}
\usepackage{graphicx} 
\usepackage{geometry}
\usepackage{multicol}
\usepackage{listings}
\usepackage{diagbox}
\usepackage{xcolor}  % 用于代码高亮颜色
\usepackage{tcolorbox} 
\tcbuselibrary{breakable} % 新增:加载 breakable 库
\usepackage{amsmath}
\usepackage{bm}  % 更好的粗体数学符号
\usepackage{wrapfig}


\lstset{
    backgroundcolor=\color{gray!10},  % 背景颜色
    basicstyle=\ttfamily\small,      % 基本字体样式
    breaklines=true,                 % 自动换行
    frame=single,                    % 边框样式
    numbers=left,                    % 显示行号
    numberstyle=\tiny\color{gray},   % 行号样式
    keywordstyle=\color{red},       % 关键字颜色
    commentstyle=\color{green!40!black},  % 注释颜色
    stringstyle=\color{orange},      % 字符串颜色
    showstringspaces=false           % 不显示字符串中的空格
}
\geometry{
    left=2cm,
    right=2cm,
    top=2cm,
    bottom=1.6cm}
\newcommand{\dd}{\mathop{}\!\mathrm{d}}
\newcommand{\vx}{\bm{x}}
\newcommand{\vt}{\bm{t}}
% 修改:添加 breakable 选项,允许盒子跨页
\newtcolorbox{mybox}[2][]{colback=#1!5!white, colframe=#1!75!black, fonttitle=\bfseries, title={#2}, breakable}
\begin{document}

% 颜色说明:蓝色代表定义,红色代表关键定理的说明、证明,黄色代表intro或解释、说明,绿色代表题
\title{Intro to Functional Analysis -- Brief Review}
\date{}
\author{}
\maketitle

\section{度量空间}
\subsection{基本定义与性质}
\begin{mybox}[blue]{度量空间}
    设$X$是一个非空集合,定义二元函数$d: X \times X \to \mathbb{R}^+$满足:
    \begin{enumerate}
        \item \textbf{(正定性)} $d(x,y)\geq 0$ 且 $d(x,y)=0 \iff x=y$
        \item \textbf{(对称性)} $d(x,y) = d(y,x)$
        \item \textbf{(三角不等式)} $d(x,y) \le d(x,z) + d(z,y)$,$\forall x,y,z \in X$
    \end{enumerate}
    \textbf{注:同一个集合$X$上可以有多种不同的度量方式,最初级的度量方式便是“离散度量”。}
\end{mybox}

\textbf{开集、闭集的定义是数分课程的必会内容,这里不再抄写。值得注意的是集合的开/闭与度量的定义有关。}

\textbf{注:在度量空间中的点列收敛即为$d(x_n,x_0)\to0\iff x_n\to x_0$。}

\begin{mybox}[blue]{映射的连续性}
    设$X,Y$均为度量空间,映射$T:X\to Y$对于$x_0\in X$满足:对任意点列$x_n\to x_0,x_n\in X$都有$Tx_n\to Tx_0\iff$ 映射$T$在$x_0$处连续。

    \textbf{注:通常还有第二种定义,这两种定义是等价的:}$\forall\varepsilon>0, \exists\delta>0, \forall x\in B(x_0, \delta), |T(x)-T(x_0)|<\varepsilon$。

    \textbf{定义的等价性证明为数分内容,略去。}
\end{mybox}

\begin{mybox}[red]{整体角度的连续刻画}
    $T$为连续映射$\iff$(1)对任意开集$U\subset Y, T^{-1}(U)$是开集;(2)对任意闭集$U\subset Y, T^{-1}(U)$是闭集;

    \textbf{(1)的必要性证明:}设$T^{-1}x\in T^{-1}(U)$,则对$x\in U,\exists\varepsilon>0, B(x,\varepsilon)\subset U$
    
    又对于$\varepsilon$存在$\delta>0$使得$\forall y\in B(T^{-1}x,\delta), d(x, Ty)<\varepsilon\implies Ty\in B(x,\varepsilon)\subset U\implies y\in T^{-1}(U)$

    因此$B(T^{-1}x,\delta)\subset T^{-1}(U)$

    \textbf{(1)的充分性证明:}考虑开球的原像$T^{-1}B(Tx_0,\varepsilon)$(其中$\varepsilon$是任取的正数)
    
    根据条件可知它是一个开集,且可以肯定$x_0\in T^{-1}B(Tx_0,\varepsilon)\implies\exists\delta>0, B(x_0,\delta)\subset T^{-1}B(Tx_0,\varepsilon)$

    综上,$\forall\varepsilon>0, \exists\delta>0, \forall x\in B(x_0, \delta), |Tx-Tx_0|<\varepsilon$。

    \textbf{最后(1)(2)等价,这可以由取补集得到。}
\end{mybox}

\begin{mybox}[blue]{同胚}
    给定度量空间$X,Y$,若映射$f:X\to Y$满足\begin{enumerate}
        \item $f$是双射
        \item $f$是连续的
        \item $f^{-1}$是连续的
    \end{enumerate}则称$X$与$Y$是同胚的,$f$为同胚映射。

    {\color{red}\textbf{注:同胚代表着共享所有拓扑不变量,这包含了紧致性,这也是本课程最常用到的结论之一。证明将在后面给出。}}
\end{mybox}






\subsection{可分度量空间}
\textbf{稠密:为数分课要求掌握的内容,定义略去。}
\begin{mybox}[blue]{可分度量空间}
    若度量空间$A$有可数的稠密子集,则称其为可分的。
\end{mybox}
\begin{mybox}[green]{常见可分例子}
    \begin{enumerate}
        \item $C[0,1]$可分。
        
        \textbf{说明:连续函数可以用多项式逼近,而任何多项式又可以用有理多项式集合逼近,有理多项式是可数的。}

        \item $L[0,1]$可分
        
        \textbf{说明:根据实变函数的知识,$L[0,1]$ 中的可积函数可以用连续函数来逼近(在 $L^1$ 范数下);而连续函数又可以由有理系数的多项式来一致逼近,且有理系数多项式集合是可数的,因此 $L[0,1]$ 是可分的。}

        \item $l^p (1\leq p<\infty)$可分。
        
        \textbf{说明:可以用有限多项非0、每一项都是有理数的数列集合逼近,这个集合是可数的,而且包含于$l^p$}
    \end{enumerate}
\end{mybox}

\begin{mybox}[green]{怎么说明一个集合是不可分的?}
    \textbf{方法:构造一个不可数集合,其间元素两两距离大于某常数。}

    \textbf{例1:$l^{\infty}$空间,考虑集合$\{\{a_n\}|a_n=0, 1\}\subset l^{\infty}$,该集合不可数,且两两距离$\geq1\implies$ $l^{\infty}$不可分。}
    \newline

    \textbf{例2:$BC(R)$空间,考虑连续函数列$f_n(x)$,其中$f_n(x)$在$[n,n+1]$上为一个“锯齿形”,其余部分为0,再考虑其中若干个函数之和组成的集合。这个集合也不可分,且两两距离为1。}
\end{mybox}

\subsection{完备度量空间}
\begin{mybox}[blue]{完备度量空间的定义}
    若度量空间$X$中的任意柯西列$\{x_n\}$都一定收敛到$X$中某点$x_0$,则称$X$是完备的。
\end{mybox}


\subsection{紧空间}
\begin{mybox}[blue]{紧空间的定义}
    若度量空间$X$中的集合$A$满足,$A$中任意点列都存在收敛子列,且收敛于$X$中的点,则称$A$为列紧的。特别地,若收敛于$A$中的点,则称为自列紧(紧)。
\end{mybox}

\begin{mybox}[red]{紧空间的等价定义}
    在度量空间 $(X, d)$ 中,关于集合 $A \subset X$ 的以下三个命题是\textbf{等价}的:
    \begin{enumerate}
        \item \textbf{序列紧(自列紧)}:$A$ 中的任意点列都有收敛子列,且极限在 $A$ 中。(这是上一条中使用的定义)
        \item \textbf{覆盖紧(开覆盖定理)}:$A$ 的任意开覆盖都必定存在有限子覆盖。
        \item \textbf{完备且完全有界}:$A$ 作为子空间是完备的,且对于任意 $\varepsilon > 0$,$A$ 都能被有限个半径为 $\varepsilon$ 的开球覆盖(即存在有限 $\varepsilon$-网)。
    \end{enumerate}
    {\color{red}\textbf{形象:紧是拓扑意义上的有限,无论用多小的开球去覆盖都只需要有限个。它是保证最值存在、局部性质能推广到整体的核心条件。}}
\end{mybox}

\subsection{压缩映射定理}
\begin{mybox}[red]{压缩映射定理}

    设 $(X, d)$ 是完备度量空间,映射 $T: X \to X$ 满足压缩条件:存在常数 $0 \leq k < 1$,使得对任意 $x, y \in X$ 有
    $d(Tx, Ty) \leq k \cdot d(x, y)$,则 $T$ 有唯一不动点 $x^* \in X$(即 $Tx^* = x^*$);



    \noindent\textbf{证明}:任取 $x_0 \in X$,令 $x_{n+1} = T(x_n)$,$n \geq 0$,则由压缩性:$d(x_{n+1}, x_n) \leq k^n d(x_1, x_0)$。

    对 $m > n$,
    \[
        d(x_m, x_n) \leq \sum_{j=n}^{m-1} d(x_{j+1}, x_j) \leq \sum_{j=n}^{m-1} k^j d(x_1, x_0) \leq \frac{k^n}{1-k}d(x_1, x_0) \to 0 \quad (n \to \infty).
    \]
    由 $X$ 的完备性,存在 $x^* \in X$ 使 $x_n \to x^*$。

    由于 $T$ 连续(压缩映射必连续),
    \[
    T(x^*) = T(\lim_{n\to\infty} x_n) = \lim_{n\to\infty} T(x_n) = \lim_{n\to\infty} x_{n+1} = x^*.
    \]

    \textbf{唯一性证明}:若 $y^*$ 也是不动点,则
    \[
    d(x^*, y^*) = d(Tx^*, Ty^*) \leq k d(x^*, y^*),
    \]
    由于 $k<1$,必须 $d(x^*, y^*) = 0$,故 $x^* = y^*$。

\end{mybox}

\section{赋范线性空间与Banach空间}

\subsection{基本定义}

\begin{mybox}[blue]{赋范线性空间的定义}
    设 $X$ 是实数域或复数域 $\mathbb{K}$ 上的线性空间。如果对于 $X$ 中的每一个元素 $x$,都有一个实数 $||x||$ 与之对应,且满足以下三个条件:
    \begin{enumerate}
        \item \textbf{(正定性)} $||x|| \ge 0$,且 $||x|| = 0 \iff x = \mathbf{0}$;
        \item \textbf{(绝对齐次性)} $||\alpha x|| = |\alpha| \cdot ||x||$,$\forall \alpha \in \mathbb{K}$,$\forall x \in X$;
        \item \textbf{(三角不等式)} $||x + y|| \le ||x|| + ||y||$,$\forall x, y \in X$。
    \end{enumerate}
    则称 $||x||$ 为元素 $x$ 的\textbf{范数 (Norm)},称 $(X, ||\cdot||)$ 为\textbf{赋范线性空间}。

    \textbf{注:} 赋范线性空间天然是一个度量空间,其诱导度量为 $d(x, y) = ||x - y||$。范数不仅刻画了向量的“长度”,还刻画了向量之间的“距离”。所有度量空间的性质(开集、闭集、连续性、完备性)都可以直接平移过来。
\end{mybox}

\begin{mybox}[red]{Banach 空间的定义}
    如果赋范线性空间 $(X, ||\cdot||)$ 作为其诱导度量空间 $(X, d)$ 是\textbf{完备的}(即其中的任意柯西列都按范数收敛到空间内的一点),则称 $X$ 为 \textbf{Banach 空间}。

    \textbf{经典例子:}
    \begin{itemize}
        \item $\mathbb{R}^n, \mathbb{C}^n$ 配合 $p$-范数是 Banach 空间。
        \item $C[a, b]$ 配合最大值范数 $||f||_\infty = \max_{t \in [a,b]} |f(t)|$ 是 Banach 空间。
        \item $L^p[a, b]$ 和 $l^p$ ($1 \le p \le \infty$) 配合各自的 $p$-范数 都是 Banach 空间。
    \end{itemize}
\end{mybox}

\subsection{范数的等价性与有限维空间}

\begin{mybox}[blue]{等价范数}
    设 $||\cdot||_1$ 和 $||\cdot||_2$ 是线性空间 $X$ 上的两种范数。如果存在常数 $c, C > 0$,使得对任意 $x \in X$,都有:
    \[ c||x||_1 \le ||x||_2 \le C||x||_1 \]
    则称这两种范数是\textbf{等价的}。

    \textbf{重要性质:} 等价范数诱导相同的拓扑结构(即收敛序列完全相同,开集闭集完全相同,完备性相同)。
\end{mybox}

\begin{mybox}[red]{有限维赋范空间的极其重要性质}
    \begin{enumerate}
        \item \textbf{范数全等价}:有限维线性空间上的\textbf{任意}两种范数都是等价的。
        \item \textbf{必定完备}:有限维赋范线性空间一定是 Banach 空间(即有限维子空间一定是闭子空间)。
        \item \textbf{Riesz 引理与紧性}:赋范线性空间 $X$ 是有限维的 $\iff$ 其闭单位球 $B = \{x \in X \mid ||x|| \le 1\}$ 是\textbf{紧集}。
    \end{enumerate}
    
    {\color{red}\textbf{核心直观:}} Riesz 引理是无穷维空间与有限维空间的分水岭。它说明在无穷维空间中,单位球一定不紧(里面可以塞下无穷多个相互距离大于等于 $1-\varepsilon$ 的点,就像一只长满长刺的刺猬)。这也是为什么我们在无穷维空间里需要引入“紧算子”和“弱收敛”等概念的根本原因!
\end{mybox}


\section{线性算子与线性泛函}
\subsection{基本定义与性质}
\begin{mybox}[blue]{线性算子的定义}
    设 $X, Y$ 是同一个数域 $\mathbb{K}$上的线性空间。
    一个映射 $T: X \to Y$ 被称为\textbf{线性算子},如果满足:
    对任意 $x, y \in X$ 和任意标量 $\alpha, \beta \in \mathbb{K}$,都有
    \[ T(\alpha x + \beta y) = \alpha T(x) + \beta T(y) \]
    
    \textbf{注:} 
    \begin{itemize}
        \item 线性算子必然把零元映射为零元,即 $T(\mathbf{0}_X) = \mathbf{0}_Y$。
        \item \textbf{极其重要的特例:} 当目标空间 $Y = \mathbb{K}$(即映射结果是一个具体的数)时,该线性算子特称为\textbf{线性泛函 (Linear Functional)}。
    \end{itemize}
\end{mybox}


\begin{mybox}[blue]{线性算子的有界性}
    若存在常数$M$使得线性算子$T:X\to Y$对于$\forall x\in X$满足$||Tx||\leq C||x||$,则称线性算子$T$有界。
\end{mybox}

\begin{mybox}[red]{线性算子的连续和有界等价}
    \textbf{必要性证明:}若$T$连续,则对任意点列$x_n\to 0$有$Tx_n\to0$

    若不有界,则可以构造模长为1的点列$\{x_n\}$使得$||Tx_n||\geq n$。
    
    注意到点列$\{\dfrac{x_n}{n}\}\to0$但$||\dfrac{Tx_n}{n}||=1$,这与题目假设矛盾。

    \textbf{充分性证明:}若$T$有界$C$,对于点列$x_n\to0$有$||Tx_n||\leq C||x_n||\to0\implies Tx_n\to0$。
\end{mybox}

\begin{mybox}[red]{非零线性泛函连续的充要条件}
    设$X$是赋范线性空间,$f$是$X$上的非零线性泛函,则$f$连续$\iff N(f)$是$X$上的闭子集。

    \textbf{必要性证明:}取$N(f)$中收敛子列$x_n\to x_0$,则由$f$的连续性可知$f(x_n)\to f(x_0)$
    
    又$f(x_n)=0$恒成立,因此$f(x_0)=0\implies x_0\in N(f)$。

    \textbf{充分性证明:}反证法,假设$f$不连续,也就是说$f$不是有界的,则存在$||x_n||=1, |f(x_n)|\geq n$

    构造$N(f)$中的点列$y_n=\dfrac{x_1}{f(x_1)}-\dfrac{x_n}{f(x_n)}\implies f(y_n)=0\implies y_n\in N(f)$

    又因为$y_n\to\dfrac{x_1}{f(x_1)}\implies f(y_n)\to 1$,这和$f(y_n)=0$矛盾!
\end{mybox}

\begin{mybox}[red]{非零线性泛函不连续的充要条件}
    设$X$是赋范线性空间,$f$是$X$上的非零线性泛函,则$f$不连续$\iff N(f)$是$X$上的稠密子集。

    \textbf{必要性证明:}由不连续性,可以取一列$x_n\to 0$使得$|f(x_n)|\geq \varepsilon$

    对于$\forall x\in X$,考虑$x-\dfrac{x_n}{f(x_n)}f(x)\in N(f)$,注意到
    \[||x-\dfrac{x_n}{f(x_n)}f(x)-x||=\dfrac{|f(x)|}{|f(x_n)|}||x_n||\leq\dfrac{|f(x)|}{\varepsilon}||x_n|| \to0\]
    因此$\forall x\in X$,存在$N(f)$中的点列趋近于$x$。

    \textbf{充分性证明:}假设连续,$\forall x\in X$,存在$N(f)$中的一列$ x_n\to x$。
    
    则$0=f(x_n)\to f(x)\implies f=0$恒成立,因此$f$是零泛函,矛盾。
\end{mybox}








\subsection{对偶空间}
\begin{mybox}[blue]{定义:对偶空间}
对于\textbf{Banach空间}$X$,考虑$X$上的所有\textbf{连续线性泛函}$f:X\to R$,事实上可以验证这些\textbf{连续线性泛函}构成\textbf{Banach空间}(对应范数为算子范数)。

将这一Banach空间记为$Y$,则称$Y$是$X$的对偶空间,记为$Y=X^*$。
\end{mybox}

\begin{mybox}[yellow]{引入:丘砖高代中的有限维例子}
    考虑定义在Banach空间$K^n$上的\textbf{线性函数(也是泛函)}$f$,它们的解析式通式形如:
    \[\forall \vx=(x_1,x_2,\cdots, x_n)\in K^n, \: f(\vx)=\sum_{k=1}^n a_k x_k\]

    所有的这些线性函数,其实就是“取分量函数”(即函数$f_i(\vx)=x_i\:(1\leq i\leq n)$)的线性组合。
    
    也就是说,这$n$个“取分量函数”构成了$(K^n)^*$的基,因此显然$(K^n)^*$也是Banach空间。
\end{mybox}

\begin{mybox}[red]{重要性质:对偶空间是Banach空间}
    \textbf{证明:}设$X$为Banach空间,$X^*$中有一个Cauchy泛函列$f_n$

    首先构造目标的收敛泛函$f$:$\forall x\in X, ||f_m(x)-f_n(x)||\leq ||f_m-f_n||||x|| \to0$
    
    因此$\{f_n(x)\}$是Cauchy列,设其极限为$y$
    
    由此可以得到一个线性映射关系$f: x\mapsto y$,即$\forall x\in X, f_n(x)\to f(x)$

    下证明$f\in X^*$,即证明有界性:因 $\{f_n\}$ 是柯西列,故是有界列,即存在 $M$ 使得 $||f_n|| \le M$。
        \[ |f(x)| = \lim_{n\to\infty} |f_n(x)| \le \lim_{n\to\infty} ||f_n|| \cdot ||x|| \le M ||x|| \]
        故 $f$ 有界,即 $f \in X^*$。

    最后证明$f_n\to f$:因为 $\{f_n\}$ 是柯西列,$\forall \epsilon > 0, \exists N$,当 $n, m > N$ 时$||f_n - f_m|| < \epsilon$
    
    即对任意 $x$ ($||x||= 1$)有$|f_n(x) - f_m(x)| < \epsilon $,\textbf{固定 $n$,令 $m \to \infty$},得$ |f_n(x) - f(x)| \le \epsilon$
    
    取上确界,即得$||f_n - f|| = \displaystyle\sup_{||x||= 1} |f_n(x) - f(x)| \le \epsilon, \quad \forall n > N \implies f_n \to f$
    
    因此$X^*$是完备的,即为Banach空间。
\end{mybox}

\begin{mybox}[red]{经典结论:$(l^p)^*$ 与 $l^q$ 的等距同构}
    设 $1 \le p < \infty$,且 $p+q=pq$。则 $(l^p)^* \cong l^q$。

    \textbf{证明:}
    
    \textbf{第一步:构造映射(从 $l^q$ 到 $(l^p)^*$)}
    对于任意 $y = (y_n) \in l^q$,定义 $l^p$ 上的线性泛函 $f_y$:
    \[ f_y(x) = \sum_{n=1}^{\infty} x_n y_n, \quad \forall x \in l^p \]
    由 H\"older 不等式:$|f_y(x)| \le ||x||_p ||y||_q$,故 $f_y \in (l^p)^*$ 且 $||f_y|| \le ||y||_q$。
    
    进一步,根据Holder 不等式取等条件),可证 $||f_y|| = ||y||_q$。
    由此建立了\textbf{保距单射} $\Phi: y \mapsto f_y$。
    \newline

    \textbf{第二步:证明满射(从 $(l^p)^*$ 到 $l^q$)}
    任取 $g \in (l^p)^*$,我们要寻找一个 $y \in l^q$ 满足$f_y=g$。
    
    1. \textbf{确定 $y$ 的形式}:
       取 $l^p$ 的标准基 $e_k = (0,\dots,1,\dots)$,定义 $y_k = g(e_k)$。
       由此得到序列 $y = (y_1, y_2, \dots)$。

    2. \textbf{验证 $y \in l^q$}:
       考虑 $y$ 的前 $n$ 项截断。构造测试向量 $x^{(n)} = (\xi_1, \dots, \xi_n, 0, \dots)$,其中:
       \[ \xi_k = |y_k|^{q-1} \text{sgn}(y_k) \]
       那么 $||x^{(n)}||_p = (\sum_{k=1}^n |y_k|^{(q-1)p})^{1/p} = (\sum_{k=1}^n |y_k|^q)^{1/p}$。
       
       利用 $g$ 的有界性:
       \[ g(x^{(n)}) = \sum_{k=1}^n y_k \xi_k = \sum_{k=1}^n |y_k|^q \le ||g|| \cdot ||x^{(n)}||_p \]
       代入范数计算并化简(两边消去 $(\sum |y_k|^q)^{1/p}$),得:
       \[ (\sum_{k=1}^n |y_k|^q)^{1/q} \le ||g|| \]
       令 $n \to \infty$,得 $||y||_q \le ||g|| < \infty\implies y \in l^q$。

    3. \textbf{验证$f_y=g$}:
       现在我们有了 $y \in l^q$,它对应的泛函是 $f_y$。
       对于任意 $x \in l^p$,由于 $p < \infty$,有限线性组合稠密。
       $g$ 和 $f_y$ 在基向量 $e_k$ 上显然相等(由 $y_k$ 定义),故由连续性,它们在整个空间上相等。
       即 $g = f_y$。
    
    综上,$\Phi$ 是保距同构。
\end{mybox}

\begin{mybox}[green]{例:$(c_0)^*=l^1$}
    \textbf{证明思路完全同上,构造$l^1$与$(c_0)^*$之间的双射。}

    \textbf{值得注意的是,$(l^{\infty})^*\neq l^1$!!}
\end{mybox}
\begin{mybox}[red]{$L^p$空间}
    $(L^p(E))^*=L^q(E)$,其中$p+q=pq,1\leq p<+\infty$
    
    证明较复杂,涉及到许多测度论知识,故略去。

    \textbf{注:此处的$E$可以是无限测度空间,这一点和$L^p$空间的包含关系不一样!}
\end{mybox}



\subsection{算子的谱}
\begin{mybox}[blue]{算子的谱与预解集}
    设 $X$ 是复 Banach 空间,$T \in \mathcal{B}(X)$ 是有界线性算子。
    考察算子 $T_\lambda = \lambda I - T$(其中 $\lambda \in \mathbb{C}$)。
    
    \textbf{1. 预解集 $\rho(T)$:}
    如果 $T_\lambda$ \textbf{存在有界的逆算子}(即 $(\lambda I - T)^{-1} \in \mathcal{B}(X)$),那么称 $\lambda\in C$ 属于 $T$ 的预解集。
    \newline
    
    \textbf{2. 谱集 $\sigma(T)$ :}
    复平面上预解集的补集,即 $\sigma(T) = \mathbb{C} \setminus \rho(T)$。
    
\end{mybox}

\begin{mybox}[red]{谱的三种分类}
    无穷维空间的奇妙之处在于,$\lambda I - T$ 出毛病的方式有三种,因此谱集被划分为三个互不相交的部分:
    
    \begin{enumerate}
        \item \textbf{点谱 $\sigma_p(T)$:}
        $T_\lambda$ \textbf{不是单射},也就是逆映射不存在。
        
        这意味着存在非零向量 $x$ 使得 $(\lambda I - T)x = 0\implies Tx = \lambda x$。
        这就是线性代数里最熟悉的\textbf{特征值}!
        
        \item \textbf{连续谱 $\sigma_c(T)$:}
        $T_\lambda$ \textbf{是单射},也就是逆映射存在,且它的值域是\textbf{稠密的}(几乎满射)。
        
        但是它的逆算子\textbf{无界},这种现象是无穷维独有的“病态”。
        
        \item \textbf{剩余谱 $\sigma_r(T)$:}
        $T_\lambda$ \textbf{是单射},也就是逆映射存在,但它的值域\textbf{不稠密}(严重不满射)。
    \end{enumerate}

\end{mybox}

\begin{mybox}[red]{谱集和正则集的性质}

    \textbf{定理:} 有界线性算子的正则集 $\rho(T)$ 一定是复平面上的\textbf{开集}。

    \textbf{定理:} 有界线性算子的谱集 $\sigma(T)$ 一定是复平面上的\textbf{非空紧集},且谱半径 $r(T) = \displaystyle\max_{\lambda \in \sigma(T)} |\lambda| \le ||T||$。

    \textbf{证明:}
    
    \textbf{1. 证明正则集 $\rho(T)$ 是开集}
    
    任取 $\lambda_0 \in \rho(T)$,这意味着逆算子 $R(\lambda_0) = (\lambda_0 I - T)^{-1}$ 存在且有界。
    我们要证明 $\lambda_0$ 的一个小邻域内的点也都属于 $\rho(T)$。
    对于任意复数 $\lambda$,我们可以做如下代数变形:
    \[ \lambda I - T = (\lambda_0 I - T) - (\lambda_0 - \lambda)I = (\lambda_0 I - T) [ I - (\lambda_0 - \lambda)R(\lambda_0) ] \]
    记 $A = (\lambda_0 - \lambda)R(\lambda_0)$。
    只要 $\lambda$ 足够接近 $\lambda_0$,使得:
    \[ ||A|| = |\lambda - \lambda_0| \cdot ||R(\lambda_0)|| < 1 \]
    即 $|\lambda - \lambda_0| < \frac{1}{||R(\lambda_0)||}$ 时,根据 \textbf{Neumann 级数定理},算子 $(I - A)$ 是可逆的。
    
    由于 $\lambda I - T$ 是两个可逆算子 $(\lambda_0 I - T)$ 和 $(I - A)$ 的乘积,因此它也是可逆的。
    这说明 $\lambda_0$ 的一个开邻域完全包含在 $\rho(T)$ 中。故 $\rho(T)$ 是开集。
    \newline

    \textbf{2. 证明谱集 $\sigma(T)$ 是紧集(有界闭集)}
    
    \begin{itemize}
        \item \textbf{闭集:} 由集合论定义,$\sigma(T) = \mathbb{C} \setminus \rho(T)$。因为刚刚证明了 $\rho(T)$ 是开集,所以其补集 $\sigma(T)$ 必然是闭集。
        
        \item \textbf{有界性:} 考察 $|\lambda| > ||T||$ 的情况。
        此时有 $||\frac{T}{\lambda}|| = \frac{||T||}{|\lambda|} < 1$。
        我们可以把算子写成:$\lambda I - T = \lambda (I - \frac{T}{\lambda})$。
        同样利用 Neumann 级数可知 $(I - \frac{T}{\lambda})$ 可逆,因此 $\lambda I - T$ 可逆。
        这意味着:所有模长大于 $||T||$ 的 $\lambda$ 都属于正则集。
        反之,谱集 $\sigma(T)$ 必须被限制在闭圆盘 $\{\lambda \in \mathbb{C} \mid |\lambda| \le ||T||\}$ 内部。
    \end{itemize}
    
    综上所述,$\sigma(T)$ 是复平面上的有界闭集,即为紧集。
    
    \textit{(注:关于“非空性”的证明此处略去细节,仅需记住结论。)}
\end{mybox}

\subsection{Banach空间上的共轭算子}
\begin{mybox}[blue]{Banach 空间上的共轭算子定义}
    设 $X, Y$ 是 Banach 空间,$T \in B(X, Y)$ 是有界线性算子。

    对于任意 $g \in Y^*$,设$T^*g\in X^*$ 是 $X$ 上的一个线性泛函,满足:
    \[ (T^*g)(x) = g(Tx), \quad \forall x \in X \]
    则称 $T$ 的共轭算子为 $T^*: Y^* \to X^*$。
\end{mybox}

\begin{mybox}[red]{Banach 共轭算子的基本性质:保范性}
    对于上述定义的共轭算子 $T^*$,有 $T^* \in B(Y^*, X^*)$,且算子范数相等:$||T^*|| = ||T||$。

    \textbf{证明:}
    \begin{enumerate}
    \item \textbf{证明 $T^*$ 是有界线性算子(且 $||T^*|| \le ||T||$):}
    线性显然。我们来计算范数:
    对于任意 $x \in X, g \in Y^*$,利用泛函有界性和算子有界性,有:
    \[ |(T^*g)(x)| = |g(Tx)| \le ||g|| \cdot ||Tx|| \le ||g||_{Y^*} \cdot ||T|| \cdot ||x|| \]
    由于这对于所有 $x$ 成立,取上确界得到 $X^*$ 中的泛函范数:
    \[ ||T^*g||_{X^*} = \sup_{||x||=1} |(T^*g)(x)| \le ||T|| \cdot ||g||_{Y^*} \]
    这说明 $T^*$ 是有界的,且 $||T^*|| \le ||T||$。

    \item\textbf{证明 $||T|| \le ||T^*||$:}
    对于任意 $x \in X$,其像 $Tx \in Y$。
    根据 Hahn-Banach 定理(选取span$\{Tx\}$作为子空间),存在一个泛函 $g_0 \in Y^*$,满足 $||g_0|| = 1$ 且 $g_0(Tx) = ||Tx||$。
    
    把这个特别的 $g_0$ 代入共轭算子的定义中:
    \[ ||Tx|| = g_0(Tx) = (T^*g_0)(x) \le ||T^*g_0|| \cdot ||x|| \le ||T^*|| \cdot ||g_0|| \cdot ||x|| = ||T^*|| \cdot ||x|| \]
    
    根据算子范数定义,取上确界得到$||T|| \le ||T^*|| $
    \end{enumerate}
\end{mybox}

\subsection{巴拿赫空间的基本定理}
\subsubsection{Hahn-Banach定理}
\begin{mybox}[red]{实Hahn-Banach定理:将一个泛函从小的子空间延拓到整个空间中}
    设实线性空间$X$的子空间是$X_1$,$p$是定义在$X$上的次线性泛函,$f$是定义在$X_1$上的线性泛函,满足$f(x)\leq p(x),\forall x\in X_1$,\textbf{则可将$f$延拓到$X$上的线性泛函$g$使得$g|_{X_1}=f, g(x)\leq p(x), \forall x\in X$。}

    \begin{mybox}[blue]{次线性泛函}
        若定义在线性空间X上的泛函$p(x)$满足,$\forall x,y\in X, \lambda>0$有下界\begin{enumerate}
            \item $p(\lambda x)=\lambda p(x)$
            \item $p(x+y)\leq p(x)+p(y)$
        \end{enumerate}
    \end{mybox}

    \textbf{形象理解:想象有一个只能测量物体长度的尺子(线性泛函),现在你想测量物体的体积(更大空间里的线性泛函)。Hahn-Banach定理说:
            如果你能在小空间(比如一维线段)上定义某种“合理的测量规则”(满足某些条件),
            那么你就能把这个测量规则安全地(保持原有线性、有界性等)推广到整个大空间(三维空间),而不会产生矛盾。}
    \newline

    \textbf{证明:STEP 1:在$X_1$的基础上,拓展一个维度。}

    不妨设$X\neq X_1, \exists x_0\in X\setminus X_1$,考虑子空间$X_2=X_1+\text{span}\{x_0\}=\{x_1+\lambda x_0|\lambda\in R, x_1\in X_1\}$

    将泛函延拓到$X_2$上:设$f(x_0)=c$(其中$c$是人为设定的,\textbf{待定})

    即$f(x_1+\lambda x_0)=\lambda c+f(x_1)$
\newline

\textbf{STEP 2:检验延拓后的泛函是否满足条件}

首先,延拓后的泛函显然保持线性。为了使其仍被$p$控制,我们要求
\[f(x_1+\lambda x_0)=f(x_1)+\lambda c\leq p(x_1+\lambda x_0), \forall \lambda \in R, x_1\in X_1\]
\[\implies c\leq \dfrac{p(x_1+\lambda x_0)-f(x_1)}{\lambda}, \forall \lambda>0, x_1\in X_1 \text{ 且 }\: c\geq \dfrac{f(x_2)-p(x_2-\mu x_0)}{\mu}, \forall \mu>0, x_2\in X_1\]

我们需要验证这两个不等式是可以同时满足的,事实上
\[\dfrac{f(x_2)-p(x_2-\mu x_0)}{\mu}\leq \dfrac{p(x_1+\lambda x_0)-f(x_1)}{\lambda}\iff f(\dfrac{x_1}{\lambda}+\dfrac{x_2}{\mu})\leq p(\dfrac{x_1}{\lambda}+ x_0)+p(\dfrac{x_2}{\mu}-x_0)\]

注意到$p(\dfrac{x_1}{\lambda}+ x_0)+p(\dfrac{x_2}{\mu}- x_0)\geq p(\dfrac{x_1}{\lambda}+\dfrac{x_2}{\mu})\geq f(\dfrac{x_1}{\lambda}+\dfrac{x_2}{\mu})$,因此这一要求能够满足,即:

\[{\sup_{\lambda>0, x\in X}} \dfrac{f(x)-p(x-\lambda x_0)}{\lambda} \leq \inf_{\lambda>0, x\in X} \dfrac{p(x+\lambda x_0)-f(x)}{\lambda}\]

\textbf{取$c$落在这一区间里即可满足控制的要求。}
\newline

\textbf{STEP 3:使用Zorn引理完成整体延拓}

设 $\mathcal{F}$ 是所有满足以下条件的$(Y, g)$ 的集合:
\begin{itemize}
    \item $X_1 \subset Y \subset X$ 是线性子空间
    \item $g: Y \to \mathbb{R}$ 是线性泛函
    \item $g|_{X_1} = f_1$(即延拓了原来定义的 $f_1$)
    \item $g(x) \leq p(x)$ 对所有 $x \in Y$
\end{itemize}

在 $\mathcal{F}$ 上定义偏序:$(Y, g) \preceq (Y', g')$ 当且仅当 $Y \subset Y'$ 且 $g'|_Y = g$。

\begin{itemize}
    \item $\mathcal{F}$ 非空:$(X_2, f_2) \in \mathcal{F}$(这里 $f_2$ 是我们在STEP 2中构造的延拓)。
    \item 任意链有上界:设 $\{(Y_\alpha, g_\alpha)\}_{\alpha \in A}$ 是 $\mathcal{F}$ 中的链。令
    \[
    Y = \bigcup_{\alpha \in A} Y_\alpha, \quad g(x) = g_\alpha(x) \text{ 当 } x \in Y_\alpha.
    \]
    由链的定义,$g$ 是良定义的线性泛函,且满足控制条件,所以 $(Y, g) \in \mathcal{F}$ 是该链的上界。
\end{itemize}

由Zorn引理,$\mathcal{F}$ 有极大元 $(Y_{\max}, g_{\max})$。
\newline

\textbf{STEP 4:证明 $Y_{\max} = X$}

假设 $Y_{\max} \neq X$,则存在 $x_0 \in X \setminus Y_{\max}$。重复STEP 1-2的过程,可将 $g_{\max}$ 延拓到 $Y_{\max} + \text{span}\{x_0\}$,这与 $(Y_{\max}, g_{\max})$ 的极大性矛盾。

因此 $Y_{\max} = X$。令 $f = g_{\max}$,则 $f: X \to \mathbb{R}$ 是 $f_1$ 的线性延拓,且满足
\[
f(x) \leq p(x) \quad \forall x \in X.
\]

\end{mybox}

\begin{mybox}[red]{复Hahn-Banach定理}
    此时的控制改为“半范数”:(1)$p(x)\geq0$;(2)$p(x+y)\leq p(x)+p(y)$;(3)$p(\lambda x)=|\lambda|p(x)$

    \textbf{定理的证明是类似的,对实部和虚部分开讨论。作为初级笔记,主要讨论实的情况。}
\end{mybox}

\begin{mybox}[red]{推论:保范延拓}
    给定赋范空间$X$和子空间$X_1$,设$f(x)$是$X_1$上的连续线性泛函,则存在$X$上的连续线性泛函$F(x)$满足:$F|_{X_1}=f, ||F||=||f||_{X_1}$(保范数延拓)。

    \textbf{在延拓过程中,可以保持泛函的范数不变:选取$p(x)=||f||_{X_1}||x||$,容易验证其可以作为控制泛函。}

    现在应用Hahn-Banach定理,$F(x)\leq ||f||_{X_1}||x||\implies ||F||\leq ||f||_{X_1}$

    又$X_1\subset X\implies ||F||\geq ||f||_{X_1}\implies ||F||=||f||_{X_1}$
\end{mybox}

\begin{mybox}[green]{例1:任意两个不同元素都存在某个连续线性泛函进行区分}
    $\forall x_1,x_2\in X$,我们要构造线性泛函$f$使得$f(x_1)\neq f(x_2)$

    记$x_0=x_1-x_2$,构造span$\{x_0\}$上的泛函:$f(\lambda x_0)=\lambda ||x_0||$,此后将其延拓得到$X$上的泛函$F$

    注意到$F(x_0)=||x_0||=F(x_1)-F(x_2)>0\implies F(x_1)\neq F(x_2)$,则$F$为所求。
\end{mybox}

\begin{mybox}[green]{例2}
    设$X$是赋范空间,则$\forall x\in X, ||x||=\sup \{|f(x)||f\in X^*, ||f||=1\}$

    \textbf{证明:}首先对于$||f||=1$的线性泛函$f$:$|f(x)|\leq ||f||||x||=||x||$

    下构造取等:构造span$\{x\}$上的泛函:$f(\lambda x)=\lambda ||x||$,则$||f||=1$,此后将其作保范延拓得到$X$上的泛函$F$,则$F(x)=x, ||F||=1$。综上$||x||=\sup \{|f(x)||f\in X^*, ||f||=1\}$
\end{mybox}

\begin{mybox}[green]{例3:点到平面距离公式的推广}
    点$x_0$到线性泛函$f$的核空间$\ker f$的距离公式$d=\dfrac{|f(x_0)|}{||f||}$

    \textbf{证明:}考虑空间$\ker f +\text{span}\{x_0\}=\{m+\alpha x_0| \alpha\in R, m\in \text{ker}f\}$,定义该空间上的泛函$F(m+\alpha x_0)=\alpha$
    
    下计算泛函$F$的范数:注意到
    \[|\alpha|=|F(m+\alpha x_0)|= |\alpha||F(x_0+\frac{m}{\alpha})|\leq |\alpha| ||F||||x_0+\frac{m}{\alpha}||\]
    约去$\alpha$,取下确界有$||F||\geq \dfrac{1}{d}$

    另一方面,注意如下式子成立\[|F(m+\alpha x_0)|\leq \dfrac{||m+\alpha x_0||}{d}\iff d\leq ||x_0+\dfrac{m}{\alpha}||\]
    则有$||F||\leq \dfrac{1}{d}\implies ||F||=\dfrac{1}{d}$。\textbf{由Hanh-Banach定理:可以对泛函$F$作保范数延拓。}

    注意到$F$和$f$的核空间相同,因此$F(x)=\lambda f(x)$,又$F(x_0)=1\implies f(x)=f(x_0)F(x)$
    
    取范数有$||f||=\dfrac{|f(x_0)|}{d}$,即距离公式$d=\dfrac{|f(x_0)|}{||f||}$  
\end{mybox}

\begin{mybox}[green]{例4:赋范线性空间里,任意有限维子空间都有闭补空间}
    {\color{red}\textbf{注意:这里没有讲唯一,与后面的正交补空间有不同!}}

    \textbf{证明:}设$X$为赋范线性空间,其子集$Y=\text{span}\{x_1, x_2, \cdots, x_n\}$

    在$Y$上构造$n$个线性泛函$f_i(\lambda x_i)=\lambda$,计算其范数可得$||f_i||=\dfrac{1}{||x_i||}$

    将其保范延拓,可得到X上的泛函$F_i$,构造线性算子$P(x)=\sum_{i=1}^{n}f_i(x)x_i$,则$\ker P$是闭空间。

    我们断言:$X=Y\bigoplus \ker P$,事实上,$\forall x\in X, x=(x-P(x))+P(x), x-P(x)\in \ker P, P(x)\in Y$

    且$\forall y\in \ker P\bigcap Y, P(y)=0\implies y=0$,因此$X=Y\bigoplus \ker P$
\end{mybox}

\begin{mybox}[red]{几何形式:凸集分离定理}
    设 $X$ 是实赋范线性空间(或 Banach 空间),$A$ 和 $B$ 是 $X$ 中两个\textbf{不相交}的非空\textbf{凸集}。

    若 $A$ 是\textbf{开集},则存在 $X$ 上的非零连续线性泛函 $f \in X^*$ 以及实数 $\gamma \in \mathbb{R}$,使得:
    \[
        f(x) < \gamma \le f(y), \quad \forall x \in A, \forall y \in B
    \]
    
    \textbf{证明:}取$-x_0\in A-B$,记$C=A-B+x_0$是一个开的凸集,则$0\in C, x_0\notin C$。

    定义Minkowski泛函作为控制泛函:$p(x)=\inf \{b>0|x\in bC\}, x\in X$,则$p(bx_0)\geq b$

    在span$(x_0)$上定义泛函$l(bx_0)=b\leq p(bx_0)$,则在span$(x_0)$上$p$是控制次泛函。

    由Hanh-Banach定理,可以将$l$进行延拓到整个空间$X$上面。
    
    设$x\in C, C$为开集合,即$x$在$C$的内部 $\implies l(x)<1$

    设$x=y-z+x_0 \implies l(x)=l(y)-l(z)+l(x_0)<1\implies l(y)<l(z)$
    \newline

    \textbf{评注: Minkowski泛函的构造思路其实来源于常规的范数:}
    
    \textbf{对于普通范数(例如$||x||=2$),我们可以将其理解成“将单位球扩大$||x||=2$倍从而覆盖到$x$”。}

    \textbf{那么对于一般的(含有0的)凸集,我们也可以如此定义一个“广义的范数”,那么这个范数也想相当于一个泛函,称为Minkowski泛函。(也就是说范数是单位球的Minkowski泛函)}
\end{mybox}

\subsubsection{纲、Baire纲定理}
\begin{mybox}[blue]{疏集}
    若集合$A$满足\textbf{闭包}$\overline{A}$没有内点,则称其为疏集。

    \textbf{没有内点的集合不一定是疏集(比如$Q$),要加上闭集的前提。}
\end{mybox}

\begin{mybox}[blue]{纲}
    若$A=\bigcup _{n=1}^{\infty} A_n$,其中$A_n$是疏集,则$A$是第一纲的,反之则为第二纲的。
\end{mybox}

\begin{mybox}[green]{第一/二纲集的常见例子}
    \begin{enumerate}
        \item \textbf{$N$赋予欧式度量:第二纲集。}若$A\subset N$且$A\neq \emptyset$,设$x_0\in A$,则有$B(x_0,1/2)=\{x_0\}\subset A$,即$x_0$为内点,因此$N$中的疏集只有空集,因此$N$是第二纲的。
        \item \textbf{$Q$在$R$中:第一纲集。}此时单点集$\{q_n\}$是一个疏集(因为此时开球不再退化成一个点),因此$Q$可以写成可数个疏集的并。
    \end{enumerate}
\end{mybox}
\begin{mybox}[red]{Baire纲定理}
    完备度量空间是第二纲的。

    \textbf{证明:}若否,设$X=\bigcup_{n=1}^{+\infty}A_n\subset \bigcup_{n=1}^{+\infty}\overline{A_n}\subset X\implies X=\bigcup_{n=1}^{+\infty}\overline{A_n}$,其中$A_n$是疏集。
    
    不妨设$A_n$是闭的。取$x_1\in A_1, x_{n+1}\in B(x_n, \dfrac{1}{n^2})$。
    
    注意到有限个闭疏集的并仍然闭,因此$B(x_n, \dfrac{1}{n^2})\nsubseteq A_1\cup A_2\cdots\cup A_n$
    
    因此在选取$x_{n+1}$时候可以使$x_{n+1}\notin A_1\cup A_2\cdots\cup A_n$

    由此构造出点列$\{x_n\}$满足$||x_{n+1}-x_n||\leq \dfrac{1}{n^2}\implies ||x_m-x_n||\leq \dfrac{1}{n^2}+\cdots+\dfrac{1}{(m-1)^2}<\dfrac{1}{n}$

    因此$\{x_n\}$是柯西列,因此收敛于$x\in X$

    我们断言$x\notin X$,否则$x\in A_k$,而$||x_n-x||<\dfrac{1}{n}\implies n$足够大时,
\end{mybox}

\begin{mybox}[green]{例1}
设$A\subset R$是第一纲集,证明$A^c-A^c=R$

\textbf{证明:}设$x\notin A^c-A^c\implies \{x\}\cap (A^c-A^c)=\varnothing \implies (x+A^c)\bigcap A^c=\varnothing\implies x+A^c \subset A$

注意到$A$是第一纲的,因此其子集$x+A^c$是第一纲的。

而$A^c$是第二纲的,平移后依旧是第二纲的,矛盾。
\end{mybox}

\begin{mybox}[green]{例2}
设$X$是无穷维的Banach空间,证明$X$没有可数个Hamel基。

\textbf{证明:}设$X$有Hamel基$\{e_1, e_2, \cdots, e_n,\cdots\}$,设$F_n=\text{span}\{e_1, e_2, \cdots, e_n\}$,这一有限维空间是闭集。

我们断言$F_n$是疏集:$\forall x\in F_n, x+0.9\varepsilon e_{n+1}\in B(x, \varepsilon)$

而$x+0.9\varepsilon e_{n+1}\notin F_n\implies B(x, \varepsilon)\nsubseteq F_n\implies x$不是内点,因此$F_n$是疏集。

又$X=\cup _{n=1}^{+\infty} F_n$,由Baire纲定理矛盾。
\end{mybox}

\begin{mybox}[green]{例3}
设$X$是无穷维的Banach空间,闭集列$\{F_n\}$满足$X=\cup _{n=1}^{+\infty} F_n$,证明$\cup _{n=1}^{+\infty} F_n^{\circ }$在$X$中稠密。

\textbf{证明:}若否,则$\exists B(x_0, r)\cap (\cup _{n=1}^{+\infty} F_n^{\circ })=\emptyset\implies B(x_0, r)\subset (\cup _{n=1}^{+\infty} (F_n-F_n^{\circ }))$

注意到$F_n-F_n^{\circ}$是疏集,因此$\cup _{n=1}^{+\infty} (F_n-F_n^{\circ })$是第一纲集,这与$B(x_0, r)$的完备性矛盾!

\textbf{注:与本题方法相似的:}设$X$是完备度量空间,$U_n$是X中的一系列稠密子集,则$\cap _{n=1}^{+\infty} U_n$也是稠密的。
\end{mybox}



\subsubsection{开映射定理 }
\begin{mybox}[blue]{开映射的定义}
    给定赋范空间$X$和$Y$,设映射$T:X\to Y$将$X$中的任意开集映射成$Y$中的开集,则称$T$是开映射。
\end{mybox}

\begin{mybox}[red]{开映射定理}
    给定两个\textbf{Banach空间}$X$和$Y$,若线性算子$T\in B(X, Y)$是满射$\iff T$是开映射。

    \textbf{必要性证明:STEP 1:运用满射对$Y$做出刻画}

    考虑集合$\bigcup _{n=1}^{\infty}TB(0,n)\subset Y$
    
    由于$Y$是满射,任意元素必定有原像,因此$Y\subset \bigcup _{n=1}^{\infty}TB(0,n)\implies Y=\bigcup _{n=1}^{\infty}TB(0,n)$

    由于$Y$是Banach空间,由Baire纲定理:存在一个$\overline{TB(0,n)}$有内点$y$
    
    设$B(y,\delta)\subset \overline{TB(0,n)}\implies B(-y,\delta)\subset \overline{TB(0,n)}$

    \textbf{说明:这是由于对称性,$\forall z\in B(-y,\delta)\implies -z\in B(y,\delta)\implies -z\in \overline{TB(0,n)}\implies z\in \overline{TB(0,n)}$}

    因此$B(0,\delta)\subset \dfrac{1}{2}(B(y,\delta)+B(-y,\delta))\subset \overline{TB(0,n)}$
\newline

    \textbf{STEP 2:证明更小的球包含于$TB(0,n)$,去掉闭包}

    \textbf{下证明$B(0,\frac{\delta}{2})\subset TB(0,n)$:}
    \[\forall y_0\in B(0,\frac{\delta}{2})\subset\overline{TB(0,\dfrac{n}{2})}\implies \exists x_0\in B(0,\dfrac{n}{2}), ||y_0-Tx_0||<\dfrac{\delta}{4}\]
    \textbf{思想:根据前面的结论,只能得到一个逼近的值,因此需要对残差继续逼近(有点像牛顿法)}
    \[\text{对}y_1=y_0-Tx_0\in B(0,\dfrac{\delta}{4})\subset \overline{TB(0,\dfrac{n}{4})}\implies \exists x_1\in B(0,\dfrac{n}{4}),||y_1-Tx_1||<\dfrac{\delta}{4}\]

    依次构造下去,可以得到$y_m\in B(0,\dfrac{\delta}{2^{m+1}})\implies y_m\to0$

    注意到$Tx_0+Tx_1+\cdots +Tx_m=y_0-y_m$,令$m\to\infty$,可得$y_0=\displaystyle\sum_{m=0}^{\infty} Tx_m=T \displaystyle\sum_{m=0}^{\infty} x_m$

    注意到$||\displaystyle\sum_{m=0}^{\infty} x_m||\leq \displaystyle\sum_{m=0}^{\infty} ||x_m||<\displaystyle\sum_{m=0}^{\infty} \dfrac{n}{2^{m+1}}=n\implies \displaystyle\sum_{m=0}^{\infty} x_m\in B(0,n)\implies y_0\in TB(0,n)$
\newline

    \textbf{STEP 3:将这个结论推广到全空间}

    设$U\subset X$是开集,我们证明$TU$是开集。$\forall y\in TU, \text{设}\:y=Tx, \exists B(x,\varepsilon)\subset U$

    由前面所证明的,$\exists \delta>0, B(0,\delta)\subset TB(0,1)\implies B(Tx,\varepsilon\delta)\subset TB(x,\varepsilon)\subset TU$

    即$B(y,\varepsilon\delta)\subset TU$,因此证明了$TU$是开集。
\newline

    \textbf{充分性证明:}若$T$是开算子,则$TB(0,1)$是开集,且包含了$0\implies B(0,\delta)\subset TB(0,1)$

    $\forall y\in Y$,考虑$\dfrac{\delta}{2||y||}y\in B(0,\delta)\subset TB(0,1)\implies \dfrac{\delta}{2||y||}y = Tx\implies y= T(\dfrac{2||y||}{\delta}x)$,因此是满射。
\end{mybox}

\begin{mybox}[red]{推论:逆算子定理}
    设$X,Y$为\textbf{Banach空间},若$T\in B(X,Y)$是双射,则$T^{-1}\in B(Y,X)$

    \textbf{证明:}因为$T$是满射,所以$T$是开映射,$TB(0,1)$是开集,且同样含有0
    
    设$B(0,\delta)\subset TB(0,1)\implies B(0,1)\subset TB(0,\dfrac{1}{\delta})\implies T^{-1}B(0,1)\subset B(0,\dfrac{1}{\delta})$

    $\forall y\in B(0,1), y\in Y$,有$||T^{-1}y||<1=\dfrac{1}{\delta}$

    对$y$作近似归一化:令$z=\dfrac{y}{||y||+\varepsilon}\in B(0,1)\implies ||T^{-1}z||<\dfrac{1}{\delta}\implies ||T^{-1}y||<\dfrac{||y||+\varepsilon}{\delta}$

    令$\varepsilon\to0\implies ||T^{-1}||\leq \dfrac{1}{\delta}$,因此$T^{-1}$是有界的,而线性易得,因此$T^{-1}\in B(Y,X)$ 
\end{mybox}

\subsubsection{共鸣定理(一致有界定理)}
\begin{mybox}[red]{共鸣定理:讨论一致有界问题}
    设$X$是Banach空间,$Y$是赋范空间,$B(X,Y)$中的算子族$\{T_{\lambda}\}_{\lambda\in \Lambda}$满足:$\forall x\in X$,$\sup_{\lambda\in\Lambda}||T_{\lambda}x||<\infty$,则算子族范数有统一上界$||T_{\lambda}||<M$

    \textbf{理解:点点收敛$\implies$一致收敛}

    \textbf{证明:}设\[A_n=\{x\in X|||T_{\lambda}x||\leq n,\forall \lambda\in\Lambda\}\]

    由算子的连续性,易证明$A_n$为闭集,则$X=\bigcup _{n=1}^{\infty}A_n\implies \exists N, A_N$有内点(Baire纲定理)

    设$B(x_0,\delta)\subset A_N\implies B(-x_0,\delta)\subset A_N\implies B(0,\delta)\subset A_N$

    则$\forall x\in X(x\neq0), \dfrac{\delta}{2||x||}x\in B(0,\delta)\subset A_N\implies ||T_{\lambda}(\dfrac{\delta}{2||x||}x)||\leq N\implies ||T_{\lambda}x||\leq \dfrac{2N}{\delta}||x||$

    因此$||T_{\lambda}||$有一致上界$\dfrac{2N}{\delta}$.
\end{mybox}

\begin{mybox}[green]{例1:构造反例}
    \textbf{证明:存在以$2\pi$为周期的连续函数$f(x)$,其Fourier级数在某给定点不收敛。}

    \textbf{利用共鸣定理,本题不需要显式给出一个具体的反例。}

    定义完备空间为所有$T=2\pi$的连续函数,其范数为$L^{\infty}$范数。不妨证明存在一个$f(x)$在$0$处不收敛。

    定义一列泛函$F_n$,为$x(0)$的Fourier级数的部分和:
    \[F_nx = \dfrac{1}{2\pi}<x(t), 1>+\dfrac{1}{\pi}\sum_{k=1}^{n} <x(t), \cos nt> = \dfrac{1}{2\pi}\int_{-\pi}^{\pi} f(t)\dfrac{\sin((n+\frac{1}{2})t)}{\sin t} \dd t\]
    下面计算泛函$F_n$的范数(这是一个积分核,是经典例题,过程略,直接使用结论):
    \[||F_n|| = \dfrac{1}{2\pi} |\int_{-\pi}^{\pi}\dfrac{\sin((n+\frac{1}{2})t)}{\sin t} \dd t|\approx \dfrac{4}{\pi^2}\ln n\to\infty\]

    因此,若所有$T=2\pi$的连续函数Fourier级数在0处收敛,由共鸣定理可推出矛盾。

    \textbf{总结:构造算子列,然后计算算子的范数(或者估计),如果发现趋于正无穷,那么必定有一个点处无法点态收敛。}
\end{mybox}

\begin{mybox}[green]{例2:将数列问题转化为算子的构造}
    给定数列$\{\beta _n\}$,若对于任意收敛数列$\{\alpha_n\}$,$\sum_{n=1}^{+\infty} \alpha_n \beta_n <+\infty$,则级数$\sum_{n=1}^{+\infty}|\beta_n|<\infty$

    \textbf{证明:}对于收敛数列空间,定义其范数为$l^{\infty}$范数。
    
    构造泛函:$T_n x= \sum_{i=1}^{n} x_i\beta _i$(其中$x$为收敛数列空间的一点)

    由共鸣定理,泛函列$T_n$的范数有统一上界$M$。
    注意到\[|T_n x|=|\sum_{i=1}^{n} x_i\beta _i|\leq ||x||\sum_{i=1}^{n} |\beta _i|\implies ||T_n||\leq \sum_{i=1}^{n} |\beta _i|\]
    当$x=(sgn(b_1), sgn(b_2), \cdots, sgn(b_n), 0 , \cdots)$时可以取等
    
    因此$||T_n||=\sum_{i=1}^{n} |\beta _i|\leq M\implies \sum_{n=1}^{+\infty}|\beta_n|<\infty$
\end{mybox}


\subsubsection{闭图像定理}
\begin{mybox}[blue]{闭算子}
       \textbf{定义:}任取点列$\{x_n\}\subset X$,若$x_n\to x$且$Tx_n\to y$,则$y=Tx$

       \textbf{理解:}连续算子的要求是,如果点列$\{x_n\}$收敛到$x$,那么$\{Tx_n\}$也必须收敛、且收敛到$Tx$。
       
       而闭算子可以使得$\{Tx_n\}$不收敛,只要求如果收敛,则必须收敛到对的地方(即$Tx$)。
       
       {\color{red}\textbf{注意:闭算子不一定把闭集映为闭集!}}

       {\color{red}\textbf{反例:}}$T:l_1\to l_1$满足$T(x_1,x_2,\cdots, x_n)=(x_1,\dfrac{x_2}{2},\cdots, \dfrac{x_n}{n},\cdots)$,有$R(T)=\{x\in l_1|\sum_{n=1}^{+\infty}n|x_n|<\infty\}$
       
       取$R(T)$中点列$x_n=(1,\dfrac{1}{2^2},\cdots, \dfrac{1}{n^2},0,\cdots)\to x=(1,\dfrac{1}{2^2},\cdots, \dfrac{1}{n^2},\cdots)\notin R(T)$,可知$R(T)$不为闭集。

       \color{red}\textbf{注意:在一般空间里,闭算子和连续算子是不充分不必要的。当$T$的定义域不闭时,连续算子推不出闭算子,而$T$的定义域为闭集时,连续算子比闭算子强。}
    \end{mybox}


\begin{mybox}[red]{闭图像定理:讨论闭线性算子有界的充分条件}
    $X,Y$为Banach空间,算子$T:X\to Y$是闭算子,则也是连续算子。

    {\color{red}\textbf{注:这个命题的逆命题也是正确的,即Banach空间中的算子闭和连续等价。}}

    \textbf{证明:}首先定义空间$X\times Y$的范数为$||(x,y)||=||x||_X +||y||_Y$
    
    考虑其中子集(算子的图象):$G(T)=\{(x, Tx)|x\in X\}$,我们断言它是完备的。
    
    事实上,若$\{(x_n, Tx_n)\}$是柯西列,则$\{x_n\}$和$\{Tx_n\}$都是柯西列。

    设$x_n\to x, Tx_n\to y$,则由闭算子条件:$y=Tx\in Y$,因此$(x_n, Tx_n)\to (x,Tx)\in X\times Y$。

    考虑算子$K:G(T) \to X$满足$K(x, Tx)=x$,这个映射显然是双射。
    
    注意到$||K(x, Tx)||=||x||\leq ||x||+||Tx||=||(x, Tx)||\implies ||K||\leq 1$,则$K$是有界的,$K\in B(X,Y)$。
    
    因此由逆映射定理:$K^{-1}$连续。

    注意到$K^{-1}x=(x, Tx)$,取范数有$||x||+||Tx||=||K^{-1}x||\leq ||K^{-1}||||x||\implies ||Tx||\leq (||K^{-1}||-1)||x|| $
    \newline

    \begin{enumerate}
        \item {\color{red}\textbf{常见用途:证明Banach空间的算子连续。因为闭算子是比连续算子更弱的条件,通常更容易证明。}}

        \item \textbf{反例:如果去掉Banach空间的前提,该命题不再成立:}
        
        考虑$C^1[0,1]$上的求导算子$D$,易知其为线性的。
        
        若$\{x_n(t)\}\to x(t)$(即函数列一致收敛),且$x_n'(t)\to y(t)$,由数学分析知识:$y(t)=x'(t)$
        
        考虑点列$x_n(t)=t^n$,则$x_n'(t)=nt^{n-1}$不收敛(范数$\to\infty$)
    \end{enumerate}
    
\end{mybox}

\begin{mybox}[green]{例1:证明算子连续}
    Hilbert空间上的线性算子$T$满足$(x, Ty)=(Tx, y)$,则$T\in B(H)$

    \textbf{证明:}设点列$x_n\to x, Tx_n\to y$。
    
    注意到$(x_n, Tz)=(Tx_n, z)\implies (x, Tz)=(y, z)=(Tx, z)\implies y=Tx$(由于$z$任意)

    因此$T$是完备空间上的闭算子,进而是连续算子。
\end{mybox}

\subsection{弱收敛}
\begin{mybox}[blue]{弱收敛}
    设$X$是一个赋范空间,给定点列$\{x_n\}\subset X$,若$\forall f\in B(X)$有$f(x_n)\to f(x)$,则称$x_n$弱收敛于$x$,记为$ x_n \xrightarrow{w} x$
\end{mybox}

\begin{mybox}[green]{弱收敛的例子}
    考虑无穷维的Hilbert空间$H$,有一组标准正交基$\{e_n\}$
    
    考虑任意$f\in B(H)$,由(后面4.2)Riesz表示定理:$f(x)=<x,y>\implies f(e_n)=<y,e_n>$

    注意到$\displaystyle\sum_{n=1}^{+\infty}<y,e_n>^2\leq ||y||^2<\infty\implies f(e_n)=<y,e_n>\to0\implies e_n$弱收敛于0

    \textbf{但是$||e_n||=1\implies e_n\nrightarrow 0$,这便是弱收敛的一个经典例子。}
\end{mybox}

\begin{mybox}[green]{例:利用对偶性证明弱收敛}
    设 $X$ 是赋范线性空间,$\{x_n\} \subset X$。若 $x_n \xrightarrow{w} x$,则 $\{x_n\}$ 是有界序列,即 $\sup_n ||x_n|| < \infty$。

    \textbf{证明:} 视 $x_n$ 为 $X^{**}$ 中的元素。定义 $X^*$ 上的线性泛函 $g_n: X^* \to \mathbb{F}$ 为:
        \[ g_n(f) = f(x_n), \quad \forall f \in X^* \]
        由Hahn-Banach定理推论知,自然嵌入是保距的,即:
        \[ ||g_n||_{X^{**}} = ||x_n||_X \]

        对于任意固定的 $f \in X^*$,由弱收敛定义知 $f(x_n) \to f(x)$,即泛函列$\{g_n\}$点点有界。
        
        由一致有界原理,存在常数 $M$,使得:
        \[ ||x_n||_X = ||g_n|| \le M, \quad \forall n \]
\end{mybox}

\begin{mybox}[blue]{弱*收敛的定义}
    设$X$是一个赋范空间,给定点列$\{f_n\}\subset X^*$,若$\forall x\in X$有$f_n(x)\to f(x)$,则称$f_n$弱*收敛于$f$,记为$f_n \xrightarrow{w} f$
\end{mybox}

\begin{mybox}[blue]{有界算子列的强、弱和一致收敛}
设 $X, Y$ 是 Banach 空间,算子列$\{T_n\}_{n=1}^\infty \subset B(X,Y)$,$T \in B(X,Y)$。

\begin{enumerate}
    \item \textbf{一致收敛(依范数收敛)}:若
    \[
    \|T_n - T\| \to 0 \quad (n \to \infty),
    \]
    则称 $T_n$ \textbf{一致收敛}于 $T$,记作 $T_n \to T$。

    \item \textbf{强收敛(逐点收敛)}:若对任意 $x \in X$,有
    \[
    \|T_n x - T x\|_Y \to 0 \quad (n \to \infty),
    \]
    则称 $T_n$ \textbf{强收敛}于 $T$,记作 $T_n \xrightarrow{s} T$。

    \item \textbf{弱收敛}:若对任意 $x \in X$ 和任意 $f \in Y^*$($Y$ 的对偶空间),有
    \[
    f(T_n x) \to f(T x) \quad (n \to \infty),
    \]
    (即 $T_nx \xrightarrow{w} Tx$)则称 $T_n$ \textbf{弱收敛}于 $T$,记作 $T_n \xrightarrow{w} T$。
\end{enumerate}

\vspace{0.5em}
\textbf{关系}:
\[
\text{一致收敛} \ \Longrightarrow \ \text{强收敛} \ \Longrightarrow \ \text{弱收敛}.
\]
反之一般不成立。特别地:
\begin{itemize}
    \item 若 $\dim Y < \infty$,则强收敛与一致收敛等价。
    \item 若 $\dim X < \infty$,则弱收敛与强收敛等价。
\end{itemize}
\end{mybox}


\section{内积空间}
\subsection{Hilbert空间}
\subsubsection{基本定义与性质}
\begin{mybox}[blue]{内积空间的定义}
    设 $X$ 是数域 $\mathbb{K}$(实数域 $\mathbb{R}$ 或复数域 $\mathbb{C}$)上的线性空间。
    若存在映射 $\langle \cdot, \cdot \rangle: X \times X \to \mathbb{K}$,满足下列公理($\forall x, y, z \in X, \alpha, \beta \in \mathbb{K}$):
    \begin{enumerate}
        \item \textbf{共轭对称性}:$\langle x, y \rangle = \overline{\langle y, x \rangle}$;
        \item \textbf{第一变元线性}:$\langle \alpha x + \beta y, z \rangle = \alpha \langle x, z \rangle + \beta \langle y, z \rangle$;
        \item \textbf{正定性}:$\langle x, x \rangle \ge 0$,且 $\langle x, x \rangle = 0 \iff x = \mathbf{0}$。
    \end{enumerate}
    则称 $\langle \cdot, \cdot \rangle$ 为 $X$ 上的\textbf{内积},称 $(X, \langle \cdot, \cdot \rangle)$ 为\textbf{内积空间}(或准 Hilbert 空间)。
\end{mybox}

\begin{mybox}[blue]{导出的范数与 Hilbert 空间}
    \textbf{导出范数}:
    内积空间可以自然地定义范数(诱导范数):
    \[ \|x\| = \sqrt{\langle x, x \rangle} \]
    在此范数下,内积空间成为赋范线性空间。
    
    \textbf{Hilbert 空间}:
    若内积空间 $X$ 关于由内积导出的范数是\textbf{完备}的,则称 $X$ 为 \textbf{Hilbert 空间}。
\end{mybox}

\begin{mybox}[red]{内积是一个二元连续函数}
    \textbf{证明关键:}\[|<x_n, y_n>-<x,y>|\leq ||x_n-x||||y_n||+||y_n-y||||x_n||\]
\end{mybox}
\begin{mybox}[red]{内积空间和赋范空间的关系}
    \begin{itemize}
    \item 当内积被定义后,范数自动由$||x||=\sqrt{<x,x>}$确定(容易验证这个范数的良定义性)。
    
    \textbf{总之:内积空间自动为赋范空间}。
    \item 而对于赋范空间,内积不一定能被定义出来。事实上,当且仅当所定义范数满足平行四边形等式
    \[||x+y||^2+||x-y||^2=2||x||^2+2||y||^2\]
    方可定义内积(极化恒等式):\[<x,y>=\dfrac{1}{4}\sum_{k=0}^{3}i^k ||x+i^ky||\]
    \end{itemize}
\end{mybox}

\begin{mybox}[green]{例1:\:$l^p$空间中只有$l^2$可以定义内积}
    \textbf{证明:}注意到\[||e_1||=||e_2||=1, ||e_1+e_2||=||e_1-e_2||=2^{\frac{1}{p}}\implies 2^{\frac{2}{p}+1}=4\implies p=2\]

    \textbf{注:}对于$l^2$空间,定义内积\[<a,b>=\sum_{k=1}^{+\infty}a_k\overline{b_k}\]
\end{mybox}


\subsubsection{正交补}
\begin{mybox}[blue]{正交补的定义}
    对于内积空间$H$的\textbf{任意}非空子集$M$,都能良定义其正交补为$M^{\perp}=\{x|x\perp M\}$。
\end{mybox}
\begin{mybox}[red]{正交补为闭集}
    对于内积空间$H$的任意非空子集$M$,其正交补$M^{\perp}$均为闭集。

    \textbf{证明:}任取$\{x_n\}\subset M^{\perp}, x_n\to x$,则$\forall y\in M, <x_n, y>=0\implies<x,y>=0$(由于内积的连续性)
\end{mybox}
\begin{mybox}[red]{正交补的完美性质}
    对于内积空间$H$的任意非空子集$M$有$M^{\perp}=\overline{\text{span}(M)}^{\perp} $

    \textbf{证明:}首先$\forall x\in M^{\perp}, \forall y\in \overline{\text{span}(M)}$,存在点列$\{y_n\}\to y, y_n\in \text{span}(M)$
    
    则有$<y_n,x>=0\implies <y,x>=0$(内积连续性)$\implies x\perp \overline{\text{span}(M)}^{\perp}$。
    
    另一方面$\forall x\in \overline{\text{span}(M)}^{\perp}$,显然有$x\perp M$(因为$M\subset \text{span}(M)$)$\implies x\in M^{\perp}$。综上$M^{\perp}=\overline{\text{span}(M)}^{\perp}$
\end{mybox}

\subsubsection{投影定理与正交分解}
\begin{mybox}[red]{最佳逼近}
    设$H$是\textbf{Banach}空间,$H_1$是其{\color{red}\textbf{闭凸子集}},则$\forall x\in H$,在$H_1$中\textbf{存在唯一}的最佳逼近元。

    \textbf{证明:}$x\in H_1$是trivial的,不妨设$x\notin H_1$,则$\{d|d=||x-y||, y\in H_1\}$有下界,设下确界为$d$。

    根据下确界的定义,存在点列$\{y_n\}\subset H_1$使得$||x-y_n||\in [d,d+\dfrac{1}{n})$

    为了证明点列收敛,只需验证其完备性:注意到$||x-\dfrac{y_m+y_n}{2}||^2\geq d^2$(由于凸)
    \[\implies 2(||x-\dfrac{y_m+y_n}{2}||^2+||\dfrac{y_m-y_n}{2}||^2)=||x-y_m||^2+||x-y_n||^2\]
    \[\implies 2d^2+\dfrac{1}{2}||y_m-y_n||^2\leq (d+\dfrac{1}{m})^2+(d+\dfrac{1}{n})^2\implies ||y_m-y_n||^2\leq 2(\dfrac{2d}{m}+\dfrac{2d}{n}+\dfrac{1}{m^2}+\dfrac{1}{n^2})\to0\]

    结合$H$的完备性,可得这个点列收敛,进而极限点在$H_1$内(由闭性)

    \textbf{唯一性证明:}若有两个最佳逼近元$y_1,y_2$,则\[\implies 2||x-\dfrac{d_1+d_2}{2}||^2+\dfrac{||y_1-y_2||^2}{2}||^2=||x-y_1||^2+||x-y_2||^2=2d^2\geq 2d^2+\dfrac{||y_1-y_2||^2}{2}\implies y_1=y_2\]
\end{mybox}

\begin{mybox}[red]{投影定理}
    设$H$是\textbf{Hilbert}空间,$H_1$是$H$的一个{\color{red}\textbf{闭子空间}},则$H=H_1\oplus H_1^{\perp}$(唯一正交分解!)。

    \textbf{注:}
    \begin{itemize}
        \item \textbf{这里的“闭子空间”比上面的“闭凸子集”强。}
        \item $\forall x\in H$,可以由此唯一分解$x=y+(x-y)$,那么这个$y$即为所求的唯一逼近元。
    \end{itemize}
    \textbf{证明:}由最佳逼近的结论,存在唯一最佳逼近元$y\in H_1$。

    由最佳可知:\[\forall \alpha\in C, z\in H_1, ||x-y-\alpha z||\geq ||x-y||\]
    \[\implies |\alpha|^2||z||^2\geq 2Re(\alpha<z,x-y>)\]
    若$<z,x-y>\neq0$,取$\alpha=\lambda\overline{<z,x-y>}$代入有$||z||\geq\sqrt{\dfrac{2}{\lambda}}$,这与$\lambda$的任意性矛盾! 

    因此$<x-y,z>=0\implies x-y\perp H_1$,由此我们证明了唯一分解,进而证明了$H=H_1\oplus  H_1^{\perp}$
\end{mybox}

\begin{mybox}[green]{例:闭空间取两次正交补等于自身}
    对于内积空间$H$的任意非空子集$M$,有$(M^{\perp})^{\perp}=\overline{\text{span}(M)}$

    \textbf{证明:}由正交补的性质,有$M^{\perp}=\overline{\text{span}(M)}^{\perp} $

    而$\overline{\text{span}(M)}$是闭空间,因此$H=\overline{\text{span}(M)}\oplus \overline{\text{span}(M)}^{\perp}\implies (\overline{\text{span}(M)}^{\perp})^{\perp}=\overline{\text{span}(M)}$

    因此$(M^{\perp})^{\perp}=\overline{\text{span}(M)}$
\end{mybox}

\subsubsection{标准(规范)正交基}
\begin{mybox}[blue]{标准正交系}
    设$H$是内积空间,若一族非零向量$\{e_{\alpha}\}_{\alpha\in\Lambda}$满足
    
    (1)正交性:\(\langle e_\alpha, e_\beta \rangle = 0\) 对任意 \(\alpha \neq \beta\);

    (2)规范性:\(\|e_\alpha\| = 1\) 对任意 \(\alpha \in I\)。
    则称这一族向量构成$H$的标准(规范)正交{\color{red}\textbf{系}}。
    
    \textbf{注意标准(规范)正交系不一定有可数个元素!!}

\end{mybox}

\begin{mybox}[red]{Bessel不等式}
    设$\{e_{\alpha}\}_{\alpha\in\Lambda}$是$H$的一个标准正交系,则$\forall x\in H$有:
    \[\sum_{\alpha\in\Lambda} |<x,e_{\alpha}>|^2\leq ||x||^2\]

    \textbf{证明:首先左边不一定是可求和的级数(因为不一定可数),}注意到
    \[\forall n\in N, \sum_{i=1}^{n}|<x,e_{\alpha_i}>|^2 \leq ||x||^2<+\infty\implies \forall m\in N, A_m=\{\alpha|\alpha\in\Lambda, |<x,e_{\alpha}>|\geq \dfrac{1}{m}\}\text{可数}\]

    则$\bigcup _{n=1}^{\infty}A_n=\{\alpha|\alpha\in\Lambda, <x,e_{\alpha}>\neq0\}$可数,设对应的$e_{\alpha}$依次为$e_1, e_2\cdots$

    则可得到Bessel不等式\[\sum_{\alpha\in\Lambda} |<x,e_{\alpha}>|^2=\sum_{n=1}^{+\infty}<x,e_{n}>^2\leq ||x||^2\]
\end{mybox}

\begin{mybox}[blue]{完备正交系:几个等价命题}
    若 $H$ 中的\textbf{标准}正交系 $\{e_{\alpha}\}_{\alpha\in\Lambda}$ 满足以下五个等价条件之一,则称 $\{e_{\alpha}\}_{\alpha\in\Lambda}$ 是 $H$ 的\textbf{完备正交系}(标准正交基)。
    证明过程略去。
    \begin{enumerate}
        \item \textbf{极大性}:
        正交补 $\{e_{\alpha}\}^{\perp} = \{\mathbf{0}\}$

        \item \textbf{稠密性(张成全空间)}:
        $\{e_{\alpha}\}_{\alpha\in\Lambda}$ 的有限线性组合所构成的集合在 $H$ 中是稠密的。
        即 $\overline{\text{span}\{e_{\alpha}\}} = H$。

        \item \textbf{傅里叶级数展开(万物皆可展开)}:
        空间中任意元素 $x$ 都可以由该正交系完美展开:
        \[ x = \sum_{\alpha\in\Lambda} \langle x, e_{\alpha} \rangle e_{\alpha} \]
        (注:由 Bessel 不等式可知,最多只有可数个系数非零)。

        \item \textbf{Parseval 等式(能量守恒)}:
        对于任意向量,Bessel 不等式恒取等号:
        \[ ||x||^2 = \sum_{\alpha\in\Lambda} |\langle x, e_{\alpha} \rangle|^2 \]

        \item \textbf{广义 Parseval 等式(内积保持)}:
        任意两个向量的内积,完全由它们在基底上的“坐标”乘积之和决定:
        \[ \langle x, y \rangle = \sum_{\alpha\in\Lambda} \langle x, e_{\alpha} \rangle \langle e_{\alpha}, y \rangle \]
    \end{enumerate}
\end{mybox}

\begin{mybox}[green]{例子:$L^2[0,2\pi]$}
    根据数学分析傅立叶级数知识可知,该空间上的一组标准正交基为$f_n(t)=\dfrac{1}{\sqrt{2\pi}}e^{int}$
\end{mybox}

\begin{mybox}[green]{例1}
    在$L^2[a,b]$中考虑函数集$S=\{e^{2\pi int}\}$(即周期为1的傅立叶正交系)

    (1)若$b-a\leq 1$,则$S^{\perp}=\{f(x)|f(x)=0 \: a.s.\}$(也就是说,区间长度够用)

    \textbf{证明:}将周期为$b-a$的$f(x)$延拓为周期为$1$的$g(x)=\begin{cases}
        f(x), a\leq x\leq b\\
        0, b<x\leq a+1
    \end{cases}$

    那么$g(x)$的Fourier系数为$\int_{a}^{a+1} g(t)e^{2\pi int}\dd t = \int_{a}^{b} f(t)e^{2\pi int}\dd t$

    若存在$g(x)\perp S\implies g(x)$的Fourier系数全部为$0\implies g(x)=0(a.s.)$

    因此$S^{\perp}=\{f(x)|f(x)=0 \: a.s.\}$

    (2)若$b-a>1$,则可构造出非零函数与$S$正交。

    \textbf{构造:}令$f(x)=\begin{cases}
        1, a\leq x<b-1\\
        -1, a+1\leq x <b\\
        0, \:\text{other cases}
    \end{cases}$验证发现$f(x)\perp S$
\end{mybox}

\begin{mybox}[green]{例2}
    设$\{e_n\}, \{f_n\}$是Hilbert空间中的两个标准正交系,且$\displaystyle\sum_{n=1}^{+\infty}||e_n-f_n||^2 <\infty$,证明两个正交系的完备性相互等价。

    \textbf{证明: 我们采用“完备$\iff S^{\perp}=\{0\}$”的条件来证明。}

    若$\{e_n\}$完备,而$\{f_n\}$不完备,则$\exists x\perp \{f_n\}\implies <x, f_n>=0$。

    注意到\[<x,e_n>^2 = <x, e_n-f_n>^2\leq ||x||^2 ||e_n-f_n||^2\implies ||x||^2\leq \]


\end{mybox}

\subsubsection{可分Hilbert空间}
\begin{mybox}[red]{Hilbert空间可分的充分必要条件}
    Hilbert空间$H$可分$\iff$其标准正交基至多可数。

    \textbf{必要性证明:}设$H$的可数稠密子集为$\{x_1,\cdots,x_n\cdots\}$,将其极大线性无关组提取出来后作Gram-Schimidt正交化可得标准正交\textbf{集}$\{e_n\}$。而这一正交规范集也是规范基:$\overline{\text{span}\{e_n\}}=\overline{\text{span}\{x_n\}}=H$ 

    \textbf{充分性证明:}设$\{e_n\}_{n=1}^{N}$是一组标准正交基,则可构造可数稠密集\[\{\sum_{n=1}^{N}a_ne_n| \text{Re } a_n, |\text{Im }a_n|\in Q\}\]


    {\color{red}\textbf{注:个数$N<\infty$时有$H\cong K^N$,$N=\infty$时有$H\cong l^2$}}

    这个命题可以通过构造等距同构证明:$T(x)=\{<x,e_n>\}$,映射$T:H\to K^N/l^2$
\end{mybox}












\subsection{Riesz表示定理}
\begin{mybox}[red]{Riesz表示定理的内容}
    设$f$是Hilbert空间$X$上的\textbf{连续线性}泛函,则存在\textbf{唯一的}$ y\in X$,使得$f(x)=<x, y>$
\end{mybox}

\begin{mybox}[yellow]{引入:$K^3$里的简单例子}
    考虑定义在Hilbert空间$K^3$上的\textbf{线性函数(也是泛函)}$f(\vx)=ax_1+bx_2+cx_3=<\vt, \vx>$

    其中$\vt=(a,b,c)$是平面$f(\vx)=ax_1+bx_2+cx_3=0$的法向量,也是Riesz表示定理当中的$y$。

    平面$ax_1+bx_2+cx_3=0$上的点是$f(\vx)$的零点集$\mathcal{N}(f)$,而$y$正是从$\mathcal{N}(f)^{\perp}$中寻得,这启示了一般性的证明思路。
\end{mybox}

\begin{mybox}[red]{Riesz表示定理的证明}
    \textbf{存在性证明:}首先我们断言空间$\mathcal{N}(f)=\{x|f(x)=0\}$是闭集。

    事实上,对于$\{x_n\}\subset \mathcal{N}(f), x_n\to x$,由连续性可得$0=f(x_n)\to f(x)\implies x\in \mathcal{N}(f)$

    回到原题,对空间进行直和分解:$X=\mathcal{N}(f)\bigoplus \mathcal{N}(f)^{\perp}$

    \textbf{按照intro的思路,}我们取定$x_0\in\mathcal{N}(f)^{\perp}$作为待寻找的$'y'$的方向
    
    则$f(x_0)\neq0$(由直和保证),下面对$\forall x\in X$进行直和分解:$x=(x-\alpha x_0)+\alpha x_0$

    其中$\alpha$用于保证$f(x-\alpha x_0)=0\implies\alpha=\dfrac{f(x)}{f(x_0)}$

    又结合垂直,可得$<x-\alpha x_0, \alpha x_0>=0 \implies f(x)=\dfrac{f(x_0)}{||x_0||^2}<x,x_0>=<x,\dfrac{\overline{f(x_0)}x_0}{||x_0||^2}>$

    因此待求的$y=\dfrac{\overline{f(x_0)}x_0}{||x_0||^2}$,存在性至此得证。

    \textbf{唯一性证明:}若$<x, y>=<x,z>\implies<x,y-z>=0$恒成立,取$x=y-z$即可得到$y=z$。
\end{mybox}

\begin{mybox}[green]{例1:Hilbert空间与自身等距同构}
    Riesz表示定理告诉我们,Hilbert空间上的连续线性泛函可以视为一个内积。那么我们考虑映射关系$f: X \to X^*$,将$x\in X$映为一个泛函:“与$x$做内积“($f_x(t)=<x,t>$)。

    \begin{itemize}
        \item \textbf{首先,这是一个满射:}这正是Riesz表示定理的结论。
        \item \textbf{其次,这个映射是线性、等距的:}线性显然,为了求$f$的范数,我们取$x\in X$,记对应的映射$\phi=f(x)$,则$\phi(y)=<x,y>\leq ||x||||y||$且等号可以取到。则$||\phi||=||x||$,由此证明了等距。
    \end{itemize}

    {\color{red}\textbf{综上,Hilbert空间的对偶空间与本身等距同构。}}

    \textbf{例如,在$l^p$空间中,仅有$p=2$才能定义内积,而其对偶空间仍是$l^2$;其他$l^p$空间无法定义内积,对偶空间也是完全不同的其他$l^p$空间。}
\end{mybox}

\subsection{一些特殊的有界线性算子}
\subsubsection{Hilbert空间上的共轭算子}
\begin{mybox}[blue]{Hilbert空间上的共轭算子定义}
    设$H_1,H_2$均为Hilbert空间,算子$S\in B(H_1,H_2), T\in B(H_2, H_1)$满足$\forall x\in H_1, y\in H_2$有\[<x, Ty>=<Sx, y>\]恒成立,则$S$与$T$互为共轭算子(伴随算子),记为$S=T^*, T=S^*$。

    这一定义其实在高等代数中已经出现过了,这里只是把$K^n$推广到了更一般的Hilbert空间。
\end{mybox}

\begin{mybox}[red]{共轭算子的基本性质}
    设$H_1,H_2$均为Hilbert空间,算子$S\in B(H_1,H_2)$,则
    \begin{itemize}
        \item $S^*$存在且唯一。
        \item $||S||=||S^*||$($K^n$版本就是,矩阵和它的共轭转置有完全相同的特征值)。
    \end{itemize}

    \textbf{证明:}设$x\in H_1, y\in H_2$,内积$<Sx, y>$可以视为定义在$H_1$上的泛函$f(x)=(Sx, y)$
    
    \textbf{我们断言$f$是一个有界线性泛函。}事实上,$f(\lambda_1x_1+\lambda_2x_2)=(S(\lambda_1x_1+\lambda_2x_2), y)=\lambda_1(Sx_1, y)+\lambda_2(Sx_2, y)=\lambda_1f(x_1)+\lambda_2f(x_2)$,因此是线性的。

    且$|f(x)|\leq ||Sx||||y||\leq ||S||||x||||y||$,因此$||f||\leq ||S||||y||$有界。

    \textbf{由Riesz表示定理,}存在\textbf{唯一的}$ z\in H_1$使得$f(x)=<x,z>$
    
    唯一性确保我们可以良定义算子$T:H_2\to H_1$使得$Ty=z$。验证$T$是有界线性算子是trivial的。
    \newline

    \textbf{下证明$||T||=||S||$:}在等式$<x, Sy>=<Tx, y>$中令$y=Tx$
    
    有$||Tx||^2=<x, STx>\leq ||x||||STx||\leq ||Tx||||S||||x||\implies ||S||\geq\dfrac{||Tx||}{||x||}$恒成立。

    取上确界有$||S||\geq ||T||$,同理有$||T||\geq ||S||$,因此$||S||=||T||$。
\end{mybox}

\begin{mybox}[red]{高等代数($K^n$)其他的性质推广:}
    设$T\in B(H)$,则有$||TT^*||=||T||^2$。

    \textbf{证明:方法完全同上。注意到}\[<x, T^*Tx>=<Tx, Tx>=||Tx||^2\leq ||x||||T^*T||||x||\implies||T^*T||\geq\dfrac{||Tx||^2}{||x||^2}\implies ||T^*T||\geq||T||^2\]

    另一方面,\[||T^*Tx||\leq ||T^*||||T||||x||=||T||^2||x||\implies ||T||^2\geq\dfrac{||T^*Tx||}{||x||}\implies ||T^*T||\leq ||T||^2\]

    因此\[||T^*T||=||T||^2\]
\end{mybox}

\begin{mybox}[red]{值域和零空间的推广}
    在线性代数里,我们有Ker$\phi \perp$ Im $\phi^*$,推广到一般Hilbert空间也有类似的结论:
    
    设$H$为Hilbert空间,$T\in B(H)$,记$R(T)$为值域,$N(T)$为零空间,则有\[N(T)=R(T^*)^{\perp}\]

    \textbf{证明:}$\forall x\in N(T), y\in R(T^*)$,设$y=T^*z$则有$<x, y>=<x, T^*z>=<Tx, z>=0\implies x\in R(T^*)^{\perp}$

    另一方面,$\forall y\in R(T^*)^{\perp}$有$0=<y,T^*Ty>=<Ty, Ty>=||Ty||^2\implies Ty=0\implies y\in N(T)$

    综上有$N(T)=R(T^*)^{\perp}$。
    \newline

    \textbf{进一步的,再取一次正交补可得}$\overline{R(T^*)}=N(T)^{\perp}$(注意有界线性算子的零空间一定闭,但值域不一定闭,所以要加闭包!)
\end{mybox}

\subsubsection{自伴算子}
\begin{mybox}[blue]{自伴算子的定义}
    $H$为Hilbert空间,$T\in B(H)$,则当$T=T^*$时称其为自伴算子。在线性代数里,此即为Hermite阵。
\end{mybox}

\begin{mybox}[red]{自伴算子的性质}
    $T$为自伴算子$\iff \forall x\in H, <x,Tx>\in R$

    \textbf{证明:}必要性显然,对于充分性:注意到\[<\lambda x+y, T(\lambda x+y)>=|\lambda|^2<x,Tx>+\lambda <x,Ty>+\overline{\lambda}<y,Tx>+<y,Ty>\in R\]
    \[\implies \lambda <x,Ty>+\overline{\lambda}<y,Tx>\in R\]

    分别取$\lambda=1, i$可推出$<T^*x, y>=<x,Ty>=\overline{<y,Tx>}=<Tx, y>$
    
    $\implies T^*x=Tx$(这是因为$y$的任意性)。
\end{mybox}

\begin{mybox}[red]{自伴算子的范数公式}
    若$T$是自伴算子,则$||T||=\displaystyle\sup_{||x||=1}|<Tx, x>|$

    \textbf{证明:}记上确界$\displaystyle\sup_{||x||=1}|<Tx, x>|$
    
    一方面,$|<Tx, x>|\leq ||Tx||||x||\leq ||T||||x||^2=||T||\implies M\leq ||T||$

    另一方面,由极化恒等式:
    \[<T(x+y), x+y> - <T(x-y), x-y> = 2<Tx, y> +2<Ty, x>= 4<Tx, y>\]

    由$M$的定义:\[4|<Tx, y>|\leq M(||x+y||^2+||x-y||^2)=2M(||x||^2+||y||^2)\]

    取$y=\lambda Tx$(参数待定)有$2\lambda ||Tx||^2 \leq M(||x||^2 +\lambda^2 ||Tx||^2)$,取$\lambda =\frac{1}{||Tx||}\implies 2||Tx||\leq M(||x||^2+1)$

    对于单位向量$x$,代入有$||Tx||\leq M\implies ||T||\leq M\implies ||T||=M$
\end{mybox}

\subsubsection{正常算子和酉算子}
\begin{mybox}[blue]{正常算子的定义}
    设$H$为Hilbert空间,若算子$T\in B(H)$满足$T^*T=TT^*$,则称$T$为正常算子。在线性代数里,这就是正规矩阵。

    \textbf{特别地,}当$T^*T=TT^*=I$时$T$为酉算子。
    
    \textbf{注:只有一个等于$I$不能推出酉算子,例如左、右移算子。}
\end{mybox}

\begin{mybox}[red]{正常算子的充要条件}
    算子$T\in B(H)$为正常算子$\iff ||Tx||=||T^*x||, \forall x\in H$

    \textbf{证明:}注意到\begin{align*}
        ||Tx||=||T^*x||&\iff <x, T^*Tx>=<x, TT^*x>\\
                    &\iff <x, (T^*T-TT^*)x>=0\\ 
                    &\iff ||T^*T-TT^*||=0\\
                    &\iff TT^*=T^*T\\
                    &\iff T\text{是正常算子}
    \end{align*}
\end{mybox}

\begin{mybox}[red]{酉算子的充要条件}
    算子$T\in B(H)$为酉算子$\iff ||Tx||=||x||, \forall x\in H$

    \textbf{证明:}注意到\begin{align*}
        ||Tx||=||x||&\iff <x, T^*Tx>=<x, x>\\
                    &\iff <x, (T^*T-I)x>=0\\ 
                    &\iff ||TT^*-I||=0, ||T^*T-I||=0\\
                    &\iff TT^*=T^*T=I\\
                    &\iff T\text{是酉算子}
    \end{align*}
\end{mybox}

\subsubsection{投影算子}
\begin{mybox}[blue]{投影算子的定义}
    设$H$为一个Hilbert空间,$H_1$是其中的一个闭空间,则有正交分解$H=H_1\bigoplus H_1^{\perp}$

    相应地,对于$x\in H$,存在唯一分解$x=x_1+x_2, x_1\in H_1, x_2\in H_1^{\perp}$。
    
    由此可定义\textbf{投影在$H_1$上的投影算子$P$使得$Px=x_1$。}
\end{mybox}

\begin{mybox}[red]{投影算子的充要条件1}
    设$H$为一个Hilbert空间,$T\in B(H)$为投影算子$\iff T$是幂等且自伴的。(线代版本表述完全相同)

    \textbf{必要性证明:}设$T$是投影到$H_1$上的算子,$\forall x\in H$分解为$x=x_1+x_2$
    
    则$Tx=x_1, T^2x=Tx_1=x_1=Tx$,由于$x$的任意性知$T^2=T$,即$T$是幂等的。
    
    另一方面,$\forall y\in H$,设其分解为$ y=y_1+y_2$,则\[<Tx, y>=<x_1, y_1+y_2>=<x_1, y_1>, <x, Ty>=<x_1+x_2, y_1>=<x_1,y_1>\implies T\text{是自伴算子}\]

    \textbf{充分性证明:}首先断言$I-T$是一个有界线性算子:注意到
    \begin{align*}
        <x-Tx, x-Tx> &= <x,x>+<Tx, Tx>-2<Tx, x>\\
        &=||x||^2+||Tx||^2-2<T^2x, x>\\
        &=||x||^2+||Tx||^2-2<Tx, Tx>\\
        &=||x||^2-||Tx||^2\leq ||x||^2\\
        &\implies ||I-T||\leq 1
    \end{align*}
    因此,$N(I-T)$是一个有界线性泛函的零空间,是闭空间。
    
    因此,可以对全空间作正交分解$H=N(I-T)\bigoplus [N(I-T)]^{\perp}$,$T$即为$N(I-T)$上的投影算子。
\end{mybox}
 
\begin{mybox}[red]{投影算子的充要条件2}
     设$H$为一个复Hilbert空间,$T\in B(H)$为投影算子$\iff ||Tx||^2=<Tx, x>,\forall x\in H$

     \textbf{必要性证明:}$||Tx||^2 = <Tx, Tx>= <x, T^*Tx>= <x, T^2x> =<x, Tx> = <Tx, x>$

     \textbf{充分性证明:}在\textbf{复}Hilbert空间里,$<Tx, x>\in R\implies T$是自伴算子。

     因此$<Tx, x>=<Tx, Tx>= <T^2x, x>\implies <x, (T-T^2)x>=0\implies T=T^2$(这是由自伴算子的范数公式得到的)
\end{mybox}

\subsubsection{紧算子}
\textbf{真正类似于矩阵算子的准确来说应该是紧算子。}

\begin{mybox}[blue]{紧算子的定义}
    设$H$是Hilbert空间,$T$是$H$上的线性算子。若$T$将$H$内的任何有界集都映成列紧集,则称$T$为紧算子。记$H$上的紧算子全体构成$C(H)$。
\end{mybox}


\begin{mybox}[blue]{有限秩算子}
    若Hilbert空间$H$中的有界线性算子$T$满足$R(T)$的维数有限,则$T$是有限维的。
    
    $H$上的有限维算子全体记为$F(H)$。
\end{mybox}

\begin{mybox}[red]{有限秩算子的本质}
    事实上,可以证明:有限秩算子$T\in F(H)$一定可以写成$Tx=\displaystyle\sum_{i=1}^{n} <x,x_i>y_i$(其中$x_i,y_i\in H$)

    \textbf{证明:}由于值域维数有限,设其标准正交基为$e_1,\cdots, e_n$,则$Tx= \displaystyle \sum_{i=1}^{n} <Tx, e_i>e_i $

    由Riesz表示定理:$f_i(x)=<Tx,e_i>$是线性泛函。
    
    因此$\exists x_i, f_i(x)=<x,x_i>\implies Tx= \displaystyle \sum_{i=1}^{n} <x, x_i>e_i$,得证。
\end{mybox}

\begin{mybox}[red]{紧算子的相关结论}
    \begin{enumerate}
        \item 紧算子是有界线性算子($C(H)\subset B(H)$)
        
        \textbf{证明:}紧算子$T$把有界集合映为列紧的,即有界的,因此$TB(0,1)$有界$M\implies ||T||\leq M$

        因此线性算子有界,即连续。

        \item $C(H)=\overline{F(H)}$
        
        \textbf{证明略去,只需要知道紧算子可以用有界线性算子逼近即可。}

        \item $T$是紧算子$\iff T^*$是紧算子。
        
        \textbf{证明:由上述2,设有限秩算子列$T_n\to T$,则$T_n^*\to T^*$且$T_n^*$也是有限秩的,因此$T^*$是紧算子。}

        \item 若$\dim H=\infty$,则任意紧算子$T\in C(H)$不存在\textbf{有界}逆算子。
        
        \textbf{证明:若存在有界的$T^{-1}$,那么$T^{-1}$连续,即$T$将开集$B(0,1)$映为开集。
        由于$0\in TB(0,1)$,因此$B(0,\delta)\subset TB(0,1)$,而紧算子要求$TB(0,1)$是列紧的。对于维数无穷的$B(0,\delta)$,其不为列紧的,矛盾。}

    \end{enumerate}
\end{mybox}

\begin{mybox}[green]{例1:紧算子的直观认识——对空间的强烈压缩}
    设$X, Y$是Banach空间,算子$T\in B(X, Y)$满足$R(T)$是闭集,且dim $R(T)=\infty$,证明$T$不是紧算子。

    \textbf{证明:}$R(T)\subset Y$为闭集,由$Y$的完备性可以得到$R(T)$的完备性。

    由开映射定理:线性算子$T:X\to R(T)$是满射,因此也是开映射。
    
    即$TB(0,1)$是$R(T)$中的开集合$\implies B(0,\delta)\subset TB(0,1)$

    而$B(0,\delta)$是一个无穷维的单位球,不是列紧的,因此$T$没有将有界集映射为列紧集,故不是紧算子。
\end{mybox}

\begin{mybox}[green]{例2:紧算子的例题}
    设$T$是Hilbert空间$H$上的一个紧算子,若$x_n\xrightarrow{w} x$,证明:$Tx_n\to Tx$

    \textbf{证明:}通过平移,不妨设$x_n\xrightarrow{w}0$,下证明$Tx_n\to0$:

    设有限秩算子列$T_m\to T$,注意到任意有限维算子$A$可以表示成$Ax =\displaystyle\sum_{i=1}^{k}<x, a_i>b_i$的形式
    
    因此,由弱收敛的定义,$ Ax_n\to 0$,且弱收敛序列$\{x_n\}$有界,不妨设上界为$C$。

    因此,$\forall m\in N, ||Tx_n||\leq ||(T-T_m)x_n||+||T_m x_n||\leq C||T-T_m||+||T_mx_n||$

    先令$n\to\infty\implies \limsup ||Tx_n||\leq C||T-T_m||$,再令$m\to\infty\implies ||Tx_n||\to0\implies Tx_n\to0$
\end{mybox}
\subsection{$L^2$空间理论的应用——条件数学期望}

\begin{mybox}[blue]{形象理解:几何视角下的条件数学期望}
    在概率论中,设 $(\Omega, \mathcal{F}, P)$ 为概率空间。
    
    所有满足 $E[X^2] < \infty$ 的随机变量构成了一个 \textbf{Hilbert 空间 $L^2(\Omega, \mathcal{F}, P)$}。

    其中的内积定义为:$\langle X, Y \rangle = E[XY]$。
\newline

    \textbf{条件期望的几何本质:}
    假设我们有一个部分信息构成的子 $\sigma$-代数 $\mathcal{G} \subset \mathcal{F}$(比如只知道某几个指标的数据)。
    基于这些部分信息能构造出的所有随机变量,构成了 $L^2$ 空间中的一个\textbf{闭子空间 $M$}($M = L^2(\Omega, \mathcal{G}, P)$)。

    现在有一个未知的随机变量 $X$(属于大空间),我们想用已知信息(也就是子空间 $M$ 里的变量 $Y$) 去\textbf{最好地预测 $X$,也就意味着均方误差 $E[(X - Y)^2]$ 最小,即在子空间 $M$ 中寻找距离 $X$ 最近的点}。
    \newline

    由 Hilbert 空间的\textbf{正交投影定理},这个距离最近的“最佳逼近点”存在且唯一,它正是 $X$ 在闭子空间 $M$ 上的\textbf{正交投影 $P_M X$}。
    在概率论中,这个正交投影被赋予了一个名字,就叫\textbf{条件数学期望}:
    \[ E[X \mid \mathcal{G}] \triangleq P_M X \]
    
    \textbf{核心推论:}
    因为是正交投影,所以误差向量 $X - E[X \mid \mathcal{G}]$ 必须垂直于子空间 $M$ 中的任意元素 $Z$。
    即内积为零:$E[(X - E[X \mid \mathcal{G}]) Z] = 0$。这正是概率论中条件期望定义的核心等式!
\end{mybox}

\end{document}